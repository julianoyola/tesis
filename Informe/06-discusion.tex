\section{Discusi�n}

La metodolog�a presentada permite la detecci�n eficaz de ampollas de la varicela, bajo las condiciones de escala descriptas. El m�todo propuesto consiste en la aplicaci�n de diferentes t�cnicas de procesamiento de im�genes, entre ellas, Canny y CHT. Se comprobaron emp�ricamente los resultados esperados utilizando fotograf�as en bruto, demostrando un excelente desempe�o.

Para poder contar con un m�todo m�s robusto se deben abordar ciertos aspectos no tratados en este trabajo. Uno de ellos consiste en poder realizar una detecci�n de las �reas de la piel, por medio de alg�n m�todo de segmentaci�n por color. Esto permitir�a poder trabajar sobre las �reas de inter�s de la fotograf�a. Existen numerosos antecedentes de trabajos de detecci�n de piel por segmentaci�n que pueden aplicarse para mejorar los resultados obtenidos hasta el momento (ver ~\cite{RA01}, ~\cite{LC01} y ~\cite{DM01}). La mayor�a de ellos trabajan sobre modelos de color como YCbCr y HSI para reconocer secciones de piel, utilizando t�cnicas de ecualizaci�n del histograma, extrapolaci�n de pixeles y filtros de suavizaci�n.

Otro aspecto a mejorar es la detecci�n de c�rculos cuando las ampollas no tienen una forma circular, por ejemplo, considerando elipses en lugar de c�rculos. Tambi�n se puede optimizar la evaluaci�n del arreglo de acumulaci�n de CHT para que pondere los votos de cada posible c�rculo contra un porcentaje de un c�rculo completo correspondiente al radio examinado, en lugar de comparar contra el m�ximo local. Asimismo, otra variante para esta evaluaci�n consiste en considerar la direcci�n del gradiente de un punto a la hora de sumar votos en la detecci�n de c�rculos (ver Rojas, Sanz, Arteaga ~\cite{TR01}).

Un aspecto adicional a considerar es la posibilidad de eliminar los falsos positivos aplicando t�cnicas de comparaci�n del color dentro y fuera del c�rculo, para determinar si las variaciones de color y luminancia corresponden a una ampolla de varicela.

% Aplicar otros algoritmos que vimos de cierre de bordes?
