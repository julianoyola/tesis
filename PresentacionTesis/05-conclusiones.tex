\section{Conclusiones}

\begin{frame}
	\frametitle{Conclusiones sobre detecci�n}
	\begin{itemize}
		\item M�todo para detecci�n de ves�culas sobre fotograf�as en bruto
		\item Espacio de color L*a*b �til para detecci�n de bordes y discriminaci�n
		\item Buen desempe�o de m�todo de Canny y CHT
	\end{itemize}
\end{frame}

\begin{frame}
	\frametitle{Conclusiones sobre discriminaci�n}
	\begin{itemize}
		\item An�lisis de discriminaci�n basado en el color interno de las ves�culas
		\item Resultados favorables en la discriminaci�n de falsos positivos
		\item Resultados poco alentadores en la discriminaci�n de varicela y herpes z�ster
		\item Test de ANOVA permite rechazar la hip�tesis nula de que las medias de varicela y herpes son iguales
		\item Variabilidad y poca cantidad de las im�genes de muestra
	\end{itemize}
\end{frame}

\begin{frame}
	\frametitle{Trabajo futuro}
	\begin{itemize}
		\item Detecci�n de piel: segmentaci�n por color
		\item Detecci�n de ampollas que no tengan forma circular
		\item An�lisis de la distribuci�n de las ves�culas
		\item Escala de las im�genes
		\item Aprendizaje autom�tico sobre el histograma del color de las ampollas detectadas
		\item Biblioteca de im�genes
		\item Comparaci�n con otras enfermedades
	\end{itemize}
\end{frame}

\begin{frame}
	\frametitle{Preguntas?}
	...
\end{frame}

\begin{frame}
	\frametitle{Gracias!}
	
	\null
	\vfill
	\scriptsize
	Virginia Arroyo (virginia.arroyo@gmail.com)\\
	Juli�n Oyola (joyola@dc.uba.ar)\\
	\textbf{Directores:} \\
	Anita Ruedin (ana.ruedin@dc.uba.ar) \\
	Daniel Acevedo (dacevedo@dc.uba.ar)
\end{frame}
