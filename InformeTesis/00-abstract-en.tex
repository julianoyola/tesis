%\begin{center}
%\large \bf \runtitle
%\end{center}
%\vspace{1cm}
\chapter*{\runtitle}

\small

\emph{
Image processing techniques can be helpful to medical professionals in the early diagnosis of skin diseases. This paper focuses on the analysis and extraction of features of varicella on digital photographs of the patient, and on the subsequent comparison to similar diseases.
The procedure used for detection consists of analyzing the luminance, enhancing the contrast by histogram equalization, smoothing the image and performing edge detection. The next step consists of applying morphological operations on the edges found, and performing the Hough transform to detect the circular form of the chickenpox vesicles. The obtained method for detecting varicella vesicles works with a reasonable rate of correct answers on a representative set of images.
After the screening, a comparative analysis of the color histograms of the vesicles is performed, focusing on numerical methods to tell real elements from false positives, and to distinguish the chickenpox vesicles from those of a similar disease like herpes zoster.
Finally, we conducted an analysis of the means of the color of the vesicles of varicella and healthy skin, as well as between vesicles of chickenpox and herpes zoster, finding encouraging differences that could be used for discrimination between healthy skin and varicella, or between chicken pox and other diseases.
}
\bigskip

\noindent\textbf{Keywords:} Circular Hough Transform, detection of vesicles of chickenpox, color representation, Gaussian filter Kullback-Leibler Divergence, Mahalanobis distance.
