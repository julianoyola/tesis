\chapter*{\runtitulo}

\small

\emph{
Las t�cnicas de procesamiento de im�genes pueden resultar de ayuda a los profesionales de la medicina en el diagn�stico temprano de enfermedades de la piel. Este trabajo se centra en el an�lisis y extracci�n de caracter�sticas propias de la varicela sobre fotograf�as digitales del paciente, y en la comparaci�n de las mismas con enfermedades similares.
El procedimiento utilizado para la detecci�n consiste en el an�lisis de la luminancia, el mejoramiento del contraste por medio de la ecualizaci�n del histograma, la suavizaci�n de la imagen y la detecci�n de bordes. Luego aplicamos operaciones morfol�gicas sobre los bordes hallados y la transformada de Hough para detectar los c�rculos de las ves�culas de la varicela. De esta forma se consigue, para un conjunto representativo de im�genes, un m�todo de detecci�n de ves�culas de la varicela con una tasa razonable de aciertos.
Una vez aplicada la detecci�n, realizamos un an�lisis comparativo de los histogramas de color de las ves�culas, centr�ndonos en m�todos num�ricos que permitan distinguir elementos reales de falsos positivos, y tambi�n distinguir ves�culas de varicela de las de otra enfermedad, como el herpes z�ster, obteni�ndose resultados altamente satisfactorios.
Por �ltimo, realizamos un an�lisis sobre las medias de los colores de las ves�culas de varicela y piel sana, como as� tambi�n entre ves�culas de varicela y herpes z�ster, encontrando diferencias alentadoras que podr�an ser utilizadas para la discriminaci�n entre piel sana y varicela, o entre varicela y otra enfermedad. 
}

\paragraph{Palabras clave:}

Transformada de Hough circular, detecci�n de ves�culas de varicela, filtro gaussiano, espacios de color, Kullback-Leibler Divergence, distancia Mahalanobis.



