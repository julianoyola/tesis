\section{Discriminaci�n de enfermedades analizando los colores de las ves�culas}

Analizamos los colores de las ves�culas de las enfermedades estudiadas: varicela y herpes z�ster. El objetivo de este an�lisis es detectar caracter�scas que indiquen diferencias entre una enfermedad y otra. Nos centramos en el estudio de las medias de los colores de las ves�culas de cada una de las enfermedades, compar�ndolas. El procedimiento que llevamos a cabo fue el siguiente. Primero tomamos cuatro im�genes representativas, dos de varicela y dos herpes z�ster y construimos una clase con las ves�culas de cada imagen. Las im�genes que utilizamos son las que se muestran en la figura (poner figura). Al realizar las pruebas encontramos que para la componente a* las diferencias son mayores si comparamos las medias de las ves�culas de varicela y herpes z�ster, que si comparamos entre ves�culas de la misma enfermedad. En el siguiente cuadro podemos ver este comportamiento. 
    
Comparasi�n entre clases:\\
$\mu_{(a)}- \mu_{(c)} =  9.1967      \in [   8.8831    , \      9.5104    ]$;  
$\mu_{(a)}- \mu_{(d)} =   6.1135      \in [  5.8445     , \      6.3826    ]$.  
$\mu_{(b)}- \mu_{(c)} =  6.5746      \in  [   6.2689    , \      6.8802     ]$;  
$\mu_{(b)}- \mu_{(d)} =   3.4913       \in [ 3.2317      , \     3.7509    ]$. 


Luego, asumiendo una distribuci�n normal de los valores y varianzas iguales, utilizamos una hip�tesis nula en la cual presumimos que las medias de os colores de las ves�culas de varicela y herpes z�ster son iguales. A trav�s de una prueba estad�stica, refutamos esta hip�tesis. Para esta prueba seguimos utilizando la componente a* de color, pero ahora separamos llas ves�culas en dos clases, una de ellas contiene los valores de las ves�culas de varicela y la otra,las ves�culas de herpes z�ster. Encontramos que la diferencia $\mu_{\scriptsize\textrm{(chickenpox)}}- \mu_{\scriptsize\textrm{(herpeszoster)}} =5.8608$, con una confianza del 100\% en el intervalo [5.7044,  6.0171].

Con estos resultados podemos concluir que existen suficiente evidencia estad�stica para refutar la hip�tesis nula en la cual presumimos que las medias de las dos clases diferentes son iguales. Adem�s, vemos que la diferencia entre clases en mayor que la diferencia en medias de la  misma enfermedad.

Realizamos pruebas similares con las componentes Cb y Cr, del modelo Y*Cb*Cr. 

Encontramos que para la componente Cb* la diferencia $\mu_{\scriptsize\textrm{(chickenpox)}}- \mu_{\scriptsize\textrm{(herpeszoster)}} =-6.4700$, con una confianza del 100\% en el intervalo [-6.6175, -6.3225].

Comparasi�n entre clases para la componenete Cb*:\\
$\mu_{(a)}- \mu_{(c)} =  -8.3343      \in [   -8.6310    , \      -8.0376    ]$;  
$\mu_{(a)}- \mu_{(d)} =   -8.0301      \in [  -8.2846     , \     -7.7756    ]$.  
$\mu_{(b)}- \mu_{(c)} =  5.2527       \in  [   4.9636   , \      5.5418     ]$;  
$\mu_{(b)}- \mu_{(d)} =   4.9486       \in [ 4.7030       , \     5.1941    ]$. 

Encontramos que para la componente Cr* la diferencia $\mu_{\scriptsize\textrm{(chickenpox)}}- \mu_{\scriptsize\textrm{(herpeszoster)}} =7.9246$, con una confianza del 100\% en el intervalo [7.7563, 8.0929].

Comparasi�n entre clases para la componenete Cr*:\\
$\mu_{(a)}- \mu_{(c)} = 12.1550     \in [  11.8231    , \      12.4869  ]$;  
$\mu_{(a)}- \mu_{(d)} =   8.9064       \in [ 8.6217      , \      9.1910    ]$.  
$\mu_{(b)}- \mu_{(c)} =  -8.0765       \in  [   -8.3999     , \     -7.7532    ]$;  
$\mu_{(b)}- \mu_{(d)} =   -4.8279       \in [ -5.1026      , \    -4.5532    ]$. 

