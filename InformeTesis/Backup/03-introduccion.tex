\section{Introducci�n}

El objetivo principal de este trabajo consiste en desarrollar un m�todo capaz de detectar ampollas de varicela, y analizar sus caracter�sticas en forma comparativa con otras enfermedades. Para obtenerlo trabajamos con t�cnicas de reconocimiento de patrones. 

Entre los temas m�s importantes en el procesamiento de im�genes digitales se encuentra el reconocimiento de patrones, debido a que est� relacionado con la identificaci�n de objetos. Este tema se ha tratado con distintos enfoques y t�cnicas, como puede apreciarse en trabajos tales como el de Flores y M�ndez ~\cite{AF01} del a�o 2009, que utiliza la segmentaci�n de im�genes y la detecci�n de bordes por Canny para encontrar los bordes de una oreja, o el trabajo de Rizon et al.\ ~\cite{CH01}, que utiliza t�cnicas de segmentaci�n y CHT (Transformada Circular de Hough) para detectar el contorno de cocos en una imagen. En visi�n artificial se han desarrollado m�todos para seguimiento trayectorias utilizando la transformada de Hough y el filtrado de Canny ~\cite{JJ01}. En cuanto al reconocimiento de objetos se ha propuesto m�todos para distinguir el ojo de una persona y poder realizar la medici�n del di�metro del iris ~\cite{BC01} utilizando Canny y CHT. Por otro lado, en im�genes satelitales se presentaron publicaciones donde se explica c�mo determinar la edad geol�gica de cr�teres en Marte utilizando como principales herramientas la detecci�n de bordes (Canny) y de c�rculos (CHT)~\cite{AF01}. Finalmente podemos mencionar un sistema biom�trico de reconocimiento del iris utilizando una c�mara convencional para la captura de im�genes propuesto en el art�culo ~\cite{TC01}, que presenta un m�todo que aplica Canny y luego CHT para luego normalizar el resultado de manera tal que el mismo puede ser comparado con otra captura. 

La metodolog�a que utilizamos est� basada en la aplicaci�n de un preprocesamiento de la imagen, la detecci�n de bordes con Canny ~\cite{JC01} y de c�rculos con la Transformada Circular de Hough ~\cite{DH01} ~\cite{SJ01}, y el an�lisis del histograma de color de las ampollas. El preprocesamiento de la imagen incluye la selecci�n del espacio de color y la ecualizaci�n del histograma, para mejorar el constraste de las im�genes. Para el an�lisis de los histogramas de color trabajamos con diferentes espacios de color y diferentes medidas de similitud entre histogramas.


