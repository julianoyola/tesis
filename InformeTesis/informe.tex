\documentclass[12pt,a4paper]{article}
%\documentclass[oribibl]{llncs}
%\usepackage{subfigure}

%\documentclass[12pt]{tesis}

\usepackage{verbatim} % Para incluir comentarios que abarquen varias l�neas
\usepackage[latin1]{inputenc} % Para acentos
\usepackage[spanish]{babel} % Titulos en castellano
\usepackage{hyperref}
\usepackage{array}
\usepackage[pdftex]{graphicx}
\usepackage[font=small,format=plain,labelfont=bf,up,textfont=it,up]{caption} % Letra de los comentarios de las imagenes
%\usepackage{sidecap}

\setcounter{secnumdepth}{5} % profundidad de las secciones

\DeclareGraphicsExtensions{.pdf,.png,.jpg}

%\author{Virginia Arroyo, Juli�n Oyola \\ \normalsize \emph{Facultad de Ciencias Exactas y Naturales, Universidad de Buenos Aires}}
%\principaladviser{Ana Ruedin y Daniel Acevedo}

\title{}
\date{}

\begin{document}

%\oddsidemargin 0in %dice al compilador de Latex que el m�rgen izquierdo ser� de 1+0 pulgadas desde el borde izquierdo de la hoja
%\textwidth 7.75in %define el ancho del texto y con esto tambi�n se puede calcular el m�rgen derecho asociado
%\topmargin 0 %coloca el margen superior del texto a 1+0 pulgadas desde el inicio de la hoja.
%\headheight 0in %define el largo del texto excluyendo el encabezado y el pie de p�gina.
%\textheight 8.5in %largo del texto

\maketitle






\vspace{3cm}


\begin{center}
\thispagestyle{empty}


\

\

\

\

\hspace{-5cm} % para dvi
\special{bmp: uni.bmp x=2 y=2}


\

{\Large {\sc Tesis de Licenciatura}}\\

\

\

{\Large {\sc Departamento de Computaci�n}}\\
\vspace{2mm}
{\Large {\sc Facultad de Ciencias Exactas y Naturales}}\\
\vspace{2mm}
{\Large {\sc Universidad de Buenos Aires}}\\

\

\

 

{\Large {\bf Tema: An�lisis y extracci�n de caracter�sticas de enfermedades de la piel: su aplicaci�n en la detecci�n de varicela}}\\ 

\

\

\


{\Large Alumnos: Juli�n Oyola y Virginia Arroyo}\\
\vspace{1mm}
{\Large Directora: Dra. Ana Ruedin}\\
\vspace{1mm}
{\Large Codirector: Dr. Daniel Acevedo}\\


\end{center}

\break
\chapter*{\runtitulo}

\small

\emph{
Las t�cnicas de procesamiento de im�genes pueden resultar de ayuda a los profesionales de la medicina en el diagn�stico temprano de enfermedades de la piel. Este trabajo se centra en el an�lisis y extracci�n de caracter�sticas propias de la varicela sobre fotograf�as digitales del paciente, y en la comparaci�n de las mismas con enfermedades similares.
El procedimiento utilizado para la detecci�n consiste en el an�lisis de la luminancia, el mejoramiento del contraste por medio de la ecualizaci�n del histograma, la suavizaci�n de la imagen y la detecci�n de bordes. Luego aplicamos operaciones morfol�gicas sobre los bordes hallados y la transformada de Hough para detectar los c�rculos de las ves�culas de la varicela. De esta forma se consigue, para un conjunto representativo de im�genes, un m�todo de detecci�n de ves�culas de la varicela con una tasa razonable de aciertos.
Una vez aplicada la detecci�n, realizamos un an�lisis comparativo de los histogramas de color de las ves�culas, centr�ndonos en m�todos num�ricos que permitan distinguir elementos reales de falsos positivos, y tambi�n distinguir ves�culas de varicela de las de otra enfermedad, como el herpes z�ster, obteni�ndose resultados altamente satisfactorios.
Por �ltimo, realizamos un an�lisis sobre las medias de los colores de las ves�culas de varicela y piel sana, como as� tambi�n entre ves�culas de varicela y herpes z�ster, encontrando diferencias alentadoras que podr�an ser utilizadas para la discriminaci�n entre piel sana y varicela, o entre varicela y otra enfermedad. 
}

\paragraph{Palabras clave:}

Transformada de Hough circular, detecci�n de ves�culas de varicela, filtro gaussiano, espacios de color, Kullback-Leibler Divergence, distancia Mahalanobis.




\section {Introducci�n}

La varicela es una enfermedad causada por el virus de la varicela z�ster, un virus de la familia de los herpesvirus. La varicela es muy contagiosa y se desarrolla principalmente en ni�os. El virus se transmite generalmente por v�a a�rea, por medio de las gotas de Fl�gge (peque�as gotitas que expelen los pacientes al respirar), y tambi�n por contacto directo, aunque en menor medida. Tiene un per�odo de incubaci�n de 10 a 21 d�as y durante �l no se presentan s�ntomas hasta que aparecen las primeras lesiones en la piel en forma de peque�os granos, que se presentan de forma s�bita y evolucionan el pocas horas pasando de manchas a lesiones sobreelevadas en la piel que se conocen como p�pulas, que despu�s se llenan de l�quido formando ves�culas. Posteriormente se llenan de pus y se conocen como p�stulas que por �ltimo forman costras. Despu�s del periodo de incubaci�n, aparecen los primeros s�ntomas que son fiebre no muy alta, malestar general y unas peque�as p�pulas rojizas. La erupci�n aparece primero en el abdomen y la espalda, y luego, se propaga a casi todas las partes del cuerpo, incluidos el rostro, el cuero cabelludo, la boca, la nariz, las orejas y los genitales. Las ves�culas de la varicela suelen medir menos de un cent�metro de ancho, tienen una base roja y aparecen en tandas o brotes en el transcurso de dos a cuatro d�as. Lo habitual es que aparezcan de 3 a 4 brotes. Por lo general, la varicela es una enfermedad leve, pero puede ser grave en embarazadas, lactantes, adolescentes, adultos y personas con sistemas inmunitarios debilitados.

Luego de la infecci�n, el virus de la varicela z�ster puede quedar latente dentro de los nervios perif�ricos del cuerpo, sin causar ning�n s�ntoma. D�as o d�cadas m�s tarde, el virus puede activarse nuevamente, saliendo de las c�lulas nerviosas y formando nuevas ves�culas en la piel, en forma de anillo agrupadas a lo largo de un dermatoma (el �rea de la piel inervada por una ra�z o nervio dorsal de la m�dula espinal). A esta nueva aparici�n del virus se la conoce como herpes z�ster (o culebrilla) y suelen padecerla personas con un sistema inmunol�gico debilitado. Cuando aparece la culebrilla, s�lo afecta un lado del cuerpo, usualmente tiene el aspecto de una franja, como un cintur�n a lo largo de una �nica l�nea o filamento nervioso. El sitio donde aparece con m�s frecuencia es la espalda, en la parte superior del abdomen o en la cara. Los primeros signos del herpes z�ster son fiebre, escalofr�os, fatiga, dolor de cabeza y malestar estomacal, lo cual puede confundir a la gente, creyendo que est�n con una gripe. Estos s�ntomas habitualmente est�n acompa�ados de una sensaci�n de hormigueo, adormecimiento o dolor en un lado del cuerpo o de la cara. Muchas personas describen el dolor como una quemaz�n, pulsaciones y un dolor punzante, con puntadas agudas e intermitentes de mucho dolor. Algunas personas experimentan una picaz�n severa o molestias m�s que un verdadero dolor. Luego de varios d�as con estos s�ntomas, aparece una erupci�n en forma de franja como un cintur�n que se extiende desde la l�nea media del cuerpo hacia afuera. La erupci�n aparece como un peque�o grupo de ampollas en forma de uvas, llenas de un l�quido claro sobre una piel enrojecida. Dentro de los tres d�as posteriores a la erupci�n, las ampollas se tornan amarillas, se secan y se forman costras. El herpes z�ster no se puede transmitir a alguien que haya tenido varicela en el pasado o haya sido vacunado para evitar la infecci�n con el virus de la varicella zoster. Alguien que no haya padecido la varicela o que no se haya vacunado, puede desarrollar varicela si toma contacto con un brote de herpes z�ster. Aproximadamente, entre el 3 por ciento y el 5 por ciento de las personas infectadas con el virus de la varicela zoster padecer�n herpes z�ster en alg�n momento de sus vidas, la mayor�a de ellas despu�s de los 50 a�os de edad.

El objetivo principal de este trabajo consiste en desarrollar un m�todo capaz de detectar ves�culas de varicela, y analizar sus caracter�sticas en forma comparativa con otras enfermedades. Para obtenerlo trabajamos con t�cnicas de reconocimiento de patrones.

Entre los temas m�s importantes en el procesamiento de im�genes digitales se encuentra el reconocimiento de patrones, debido a que est� relacionado con la identificaci�n de objetos. Este tema se ha tratado con distintos enfoques y t�cnicas, como puede apreciarse en trabajos tales como el de Flores y M�ndez ~\cite{AF01} del a�o 2009, que utiliza la segmentaci�n de im�genes y la detecci�n de bordes por Canny para encontrar los bordes de una oreja, o el trabajo de Rizon et al.\ ~\cite{CH01}, que utiliza t�cnicas de segmentaci�n y la transformada circular de Hough o CHT (Circular Hough Transform), para detectar el contorno de cocos en una imagen. En visi�n artificial se han desarrollado m�todos para seguimiento trayectorias utilizando la transformada de Hough y el filtrado de Canny ~\cite{JJ01}. En cuanto al reconocimiento de objetos se ha propuesto m�todos para distinguir el ojo de una persona y poder realizar la medici�n del di�metro del iris ~\cite{BC01} utilizando Canny y CHT. Por otro lado, en im�genes satelitales se presentaron publicaciones donde se explica c�mo determinar la edad geol�gica de cr�teres en Marte utilizando como principales herramientas la detecci�n de bordes (Canny) y de c�rculos (CHT)~\cite{AF01}. Finalmente podemos mencionar un sistema biom�trico de reconocimiento del iris utilizando una c�mara convencional para la captura de im�genes propuesto en el art�culo ~\cite{TC01}, que presenta un m�todo que aplica Canny y luego CHT para luego normalizar el resultado de manera tal que el mismo puede ser comparado con otra captura. 

Para detectar las ves�culas de varicela y analizar sus caracter�sticas, aplicamos distintas metodolo g�as, de acuerdo a los resultados que deseabamos obtener. En la etapa de preprocesamiento de la imagen, trabajamos con distintos espacios de color, hasta hallar el adecuado para cada una de las etapas posteriores. Tambi�n en esta etapa debimos mejorar el contraste de las im�genes para prepararlas para la siguiente etapa. En la etapa de an�lisis de formas, probamos con varios m�todos la detecci�n de bordes, qued�ndonos con el m�todo que en la bibliografia recomienda como m�s robusto; y que luego durante las pruebas result� adecuado. El m�todo de detecci�n de bordes elegido fue el m�todo Canny ~\cite{JC01}. Adem�s en esta etapa debimos detectar la forma circular de las ves�culas de varicela. Para esta detecci�n utilizamos la Transformada Circular de Hough ~\cite{DH01} ~\cite{SJ01}. En la �ltima etapa trabajamos en la obtenci�n de caracter�sticas de las ves�culas de varicela y de herpes para poder identificar, a trav�s de esas caracter�sticas, las im�genes que contengan ves�culas de varicela o herpes, como as� tambi�n poder determinar regiones de piel sana y regiones con ves�culas. Para ello utilizamos los histogramas de color de las ves�culas en distintos espacios de color y los comparamos con diferentes m�todos, tanto sim�tricos: distancias o normas, como no sim�tricos como el KLD (Kullback Leibler divergence). Tambi�n analizamos las medias de las ves�culas de varicela y de herpes, utilizando diferentes tests, que nos permiten aproximarnos a la extracci�n de caracter�sticas que puedan ayudar en la identificaci�n de regiones de la imagen que contengan ves�culas de la enfermedad estudiada. Elegimos im�genes de herpes z�ster para comparar con las im�genes de varicela; y de esta forma evaluar como se comporta el procedimiento propuesto, dado que ambas enfermedades causan lesiones similares en la piel.

En el curso de esta tesis hemos presentado dos trabajos, una presentaci�n en un congreso nacional ~\cite{JO02} y otra en un congreso internacional ~\cite{JO01}.

A continuaci�n se describe c�mo se encuentra organizado el presente trabajo. En la secci�n 2 hablamos de las im�genes utilizadas y de sus propiedades. En la secci�n 3 detallamos los m�todos que utilizamos para detectar ves�culas (ecualizaci�n del histograma, selecci�n del espacio de color, m�todo Canny, transformada circular de Hough, entre otros). En la secci�n 4 explicamos como discriminamos entre ves�culas de varicela y ves�culas de otras enfermedades como el herpes, utilizando histogramas de las componentes de color y analiz�ndolos a trav�s de la norma y la divergencia KLD. En la misma secci�n, analizamos las medias de las componentes de color de las ves�culas de varicela y las comparamos con piel sana y con ves�culas de herpes. En la secci�n 5 damos las conclusiones finales del trabajo y posibles l�neas de investigaci�n a futuro.

\section{Las im�genes de piel y sus caracter�sticas}

\begin{frame}
	\frametitle{Caracter�sticas de las im�genes utilizadas}
	Caracter�sticas de las im�genes utilizadas
	\begin{itemize}
		\item Heterogeneidad 
		\item Escala 
		\item Ruido 
		\item Luminosidad
		\item Imperfecciones de la piel
		\item Elementos ajenos
		\item Distintas etapas de la enfermedad
	\end{itemize}
\end{frame}

\begin{frame}
	\frametitle{Ejemplo: Variabilidad de im�genes para una misma enfermedad}
	\begin{tabular}{ cc }
		\includegraphics[width=1.6in]{../Imagenes/U.Iowa/Varicela/chicken_pox_picture_13.jpg} & \includegraphics[width=1.6in]{Resources/recorte-varicella_18.jpg} \\
		\includegraphics[width=1.6in]{../Imagenes/U.Iowa/Varicela/Varicel-04.jpg} & \includegraphics[width=1.6in]{../Imagenes/U.Iowa/Varicela/Varicel-01.jpg} \\
	\end{tabular}
\end{frame}

\begin{frame}
	\frametitle{Escala}
	\begin{tabular}{cc}
		\centering
		\includegraphics[width=1.2in]{../Imagenes/U.Iowa/Varicela/chicken_pox_picture_01.jpg} &
		\includegraphics[width=2in]  {../Imagenes/U.Iowa/Varicela/Varicel-02.jpg}
	\end{tabular}
\end{frame}

\begin{frame} 
	\frametitle{T�cnicas utilizadas y medidas adoptadas}
	T�cnicas y medidas adoptadas
	\begin{itemize}
		\item Elecci�n de un subconjunto de las im�genes \pause
		\item Ecualizaci�n del histograma (Contrast-limited adaptive histogram equalization) \pause
		\item Reducci�n del ruido o suavizaci�n utilizando un filtro gaussiano \pause
		\item Elecci�n del espacio de color 
	\end{itemize}
\end{frame}

\begin{frame}
	\frametitle{Algunas im�genes seleccionadas}
	\begin{tabular}{ cc }
		\includegraphics[width=1.6in]{../Imagenes/U.Iowa/Varicela/Varicel-01.jpg} & \includegraphics[width=1.6in]{../Imagenes/U.Iowa/Varicela/Varicel-02.jpg} \\
		\includegraphics[width=1.6in]{../Imagenes/U.Iowa/Varicela/Varicel-03.jpg} & \includegraphics[width=1.6in]{../Imagenes/U.Iowa/Varicela/Varicel-04.jpg} \\
	\end{tabular}
\end{frame}

\begin{frame}
	\frametitle{Algunas im�genes seleccionadas}
	\begin{tabular}{ cc }
		\includegraphics[width=1.6in]{../Imagenes/U.Iowa/Varicela/chicken_pox_primary_lesions_03.jpg} & \includegraphics[width=1.6in]{../Imagenes/U.Iowa/Varicela/varicella_20.jpg} \\
		\includegraphics[width=1.6in]{../Imagenes/U.Iowa/Shingles/Fase1/herpes_zoster_8.jpg} & \includegraphics[width=1.6in]{../Imagenes/U.Iowa/Shingles/Fase1/herpes_zoster_114-crop.jpg} \\
	\end{tabular}
\end{frame}
\section{Detecci�n de ves�culas}

\subsection{Espacio de color}
\begin{frame}
	\frametitle{Espacio de color}
	�C�mo representamos los colores y la luz en el ordenador?
	\begin{itemize}
		\item Espacios de color posibles
		\item Luminancia vs Crominancia
		\item YUV vs L*a*b
	\end{itemize}
	Luminancia: detecci�n de bordes\\
	Crominancia: detecci�n de piel y falsos positivos
\end{frame}

\begin{frame}
	\frametitle{Luminancia vs Crominancia}
	\begin{tabular}{cc}
		\includegraphics[width=2in]{Resources/CompL-Varicel-02.jpg} & Luminancia - componente L \\
	\end{tabular}
\end{frame}

\begin{frame}
	\frametitle{Luminancia vs Crominancia}
	\begin{tabular}{cc}
		\includegraphics[width=2in]{Resources/CompA-Varicel-02.jpg} & Crominancia - componente a \\
	\end{tabular}
\end{frame}

\begin{frame}
	\frametitle{Luminancia vs Crominancia}
	\begin{tabular}{cc}
		\includegraphics[width=2in]{Resources/CompB-Varicel-02.jpg} & Crominancia - componente b \\
	\end{tabular}
\end{frame}

\subsection{Detecci�n de bordes}
\begin{frame}
	\frametitle{Detecci�n de bordes}
	\begin{itemize}
		\item Resulta sencillo para el ser humano
		\item Borde: frontera entre el objeto y el fondo
		\item Existen varios m�todos (Canny, Roberts, Sobel o Prewitt)
		\item Objetivos de un detector de borde:
		\begin{itemize}
			\item Baja tasa de error
			\item Buena localizaci�n del borde
		\end{itemize}
	\end{itemize}
\end{frame}

\begin{frame}
	\frametitle{M�todo de Canny}
	\begin{itemize}
		\item Robusto contra el ruido
		\item Gran adaptabilidad
		\item Etapas del m�todo:
		\begin{itemize}
			\item Suavizado de la imagen: Filtro gaussiano\\
			\item Obtenci�n del gradiente: Filtro pasa altos en direcci�n vertical y horizontal \\
			\item Supresi�n de puntos que no son m�ximos locales: Adelgazamiento del ancho de los bordes hasta lograr bordes de un p�xel de ancho\\
			\item Umbral con hist�resis: Funci�n de hist�resis basada en dos umbrales; con este proceso se trata de reducir la posibilidad de aparici�n de contornos falsos.\\
		\end{itemize}
	\end{itemize}
\end{frame}

\begin{frame}
	\frametitle{Operaciones morfol�gicas}
	\begin{itemize}
		\item Herramientas muy utilizadas en el procesamiento de im�genes
		\item Simplificar los datos de una imagen
		\item Preservar las caracter�sticas esenciales
		\item Eliminar aspectos irrelevantes
		\item bridge: Une pixeles que est�n separados
		\item Otras operaciones: open, close, clean
	\end{itemize}
\end{frame}

\begin{frame}
	\frametitle{Ejemplo: Bordes detectados en algunas im�genes}
	\begin{columns}[c]
	\column{1.5in}
		\includegraphics[scale=2.33]{Resources/Varicel-03.jpg} \\
		\includegraphics[scale=0.21]{Resources/Varicel-03_bordes_detectados.png} \\
		\includegraphics[scale=0.21]{Resources/Varicel-03_imagen_con_bordes.png} \\
	\column{1.5in}	
		\includegraphics[scale=2.8]{Resources/Varicel-04.jpg} \\
		\includegraphics[scale=0.2]{Resources/Varicel-04_bordes_detectados.png} \\
		\includegraphics[scale=0.2]{Resources/Varicel-04_imagen_con_bordes.png} \\
	\end{columns}	
\end{frame}

\begin{frame}
	\frametitle{Ejemplo: Bordes detectados en la imagen}
	\includegraphics[width=4.3in]{Resources/bordesEnLaImagen-Varicel-01.jpg} \\
\end{frame}

\begin{frame}
	\frametitle{Ejemplo: Bordes detectados}
	\includegraphics[width=4.3in]{Resources/bordes-Varicel-01.jpg}
\end{frame}

\subsection{Detecci�n de c�rculos}
\begin{frame}
	\frametitle{Detecci�n de c�rculos}
	\begin{itemize}
		\item �Dados los bordes, cu�ndo conforman un c�rculo?
		\item CHT: Circular Hough Transform
		\begin{itemize}
			\item Espacio de Hough
			\item Arreglo de acumulaci�n
		\end{itemize}
		\item Selecci�n de candidatos
		\begin{itemize}
			\item Ponderaci�n con respecto al m�ximo
			\item Umbralizaci�n
		\end{itemize}
	\end{itemize}
\end{frame}

\begin{frame}
	\frametitle{Ejemplo: Imagen con bordes detectados}
	\begin{figure}[h]
		\includegraphics[width=4in]{Resources/resultado-Varicel-02-radio24-bordes.jpg}
	\end{figure}
\end{frame}

\begin{frame}
	\frametitle{Ejemplo: Arreglo de acumulaci�n}
	\begin{figure}[h]
		\includegraphics[width=3.5in]{Resources/acumulador-Varicel-02.jpg}
	\end{figure}
\end{frame}

\begin{frame}
	\frametitle{Ejemplo: Imagen con el c�rculo detectado}
	\begin{figure}[h]
		\includegraphics[width=4in]{Resources/resultado-Varicel-02-radio23_1.jpg}
	\end{figure}
\end{frame}

\subsection{Falsos positivos y falsos negativos}
\begin{frame}
	\frametitle{Falsos positivos y falsos negativos}
	\begin{itemize}
		\item Detecci�n de c�rculos redundantes
		\begin{figure}[h]
			\includegraphics[width=2.2in]{Resources/resultado-Varicel-02-radio23_1.jpg}
			\includegraphics[width=2.2in]{Resources/resultado-Varicel-02-radio24.jpg}
		\end{figure}
		\item An�lisis del interior de la ampolla: Discriminaci�n
	\end{itemize}
\end{frame}

\section{Discriminaci�n entre varicela y otras enfermedades}

\subsection{Construcci�n de un modelo te�rico}
\begin{frame}
	\frametitle{Selecci�n de ves�culas de referencia}
	...
\end{frame}

\section{Conclusiones sobre la detecci�n de ves�culas}

La metodolog�a presentada permite la detecci�n eficaz de ves�culas de la varicela, bajo las condiciones de escala descriptas. El m�todo propuesto consiste en la aplicaci�n de diferentes t�cnicas de procesamiento de im�genes, entre ellas, Canny y CHT. Se comprobaron emp�ricamente los resultados esperados utilizando fotograf�as en bruto, demostrando un excelente desempe�o.

Para poder contar con un m�todo m�s robusto se deben abordar ciertos aspectos no tratados en este trabajo. Uno de ellos consiste en poder realizar una detecci�n de las �reas de la piel, por medio de alg�n m�todo de segmentaci�n por color. Esto permitir�a trabajar sobre las �reas de inter�s de la fotograf�a. Existen numerosos antecedentes de trabajos de detecci�n de piel por segmentaci�n que pueden aplicarse para mejorar los resultados obtenidos hasta el momento (ver ~\cite{RA01}, ~\cite{LC01} y ~\cite{DM01}). La mayor�a de ellos trabajan sobre modelos de color como YCbCr y HSI para reconocer secciones de piel, utilizando t�cnicas de ecualizaci�n del histograma, extrapolaci�n de pixeles y filtros de suavizaci�n.

Otro aspecto a mejorar es la detecci�n de c�rculos cuando las ves�culas no tienen una forma circular, por ejemplo, considerando elipses en lugar de c�rculos. Tambi�n se puede optimizar la evaluaci�n del arreglo de acumulaci�n de CHT para que pondere los votos de cada posible c�rculo contra un porcentaje de un c�rculo completo correspondiente al radio examinado, en lugar de comparar contra el m�ximo local. Asimismo, otra variante para esta evaluaci�n consiste en considerar la direcci�n del gradiente de un punto a la hora de sumar votos en la detecci�n de c�rculos (ver Rojas, Sanz, Arteaga ~\cite{TR01}).

\section{Conclusiones sobre la discriminaci�n de enfermedades}

El m�todo de comparaci�n de histogramas por medio de divergencia KLD contra un histograma promedio permite distinguir entre ves�culas de varicela y falsos positivos, con resultados alentadores. Este mismo m�todo no arroja conclusiones favorables para la discriminaci�n entre ves�culas de varicela y herpes z�ster. Las t�cnicas adicionales consideradas se basaron en la medici�n de la distancia de medias, con resultados ambiguos. El an�lisis de la distancia de Mahalanobis comparando ves�culas de varicela con falsos positivos de piel, da tambi�n buenos resultados que permiten discriminarlos, pero falla al aplicarse a la comparaci�n con herpes. El test de la distribuci�n de Hotelling T-Cuadrada no permite afirmar que las medias de la distribuci�n de color de varicela y herpes sean distintas.
El test de ANOVA sobre componentes de una variable de color arroja resultados m�s alentadores, permitiendo rechazar la hip�tesis nula de que las medias de varicela y herpes son iguales. Sin embargo, no alcanza para concluir que todas las ves�culas de varicela compartan la misma media.

Estos resultados pueden atribuirse a la variabilidad y poca cantidad de las im�genes de muestra con las que se cont� para el trabajo. Las condiciones variables de luz y c�mara fotogr�fica introducen diferencias en la distribuci�n de los componentes de color que alteran las medias de las distribuciones. Tampoco se puede descartar que realmente no exista diferencia apreciable a nivel num�rico entre las ves�culas de varicela y herpes z�ster, a pesar de la diferencia visible en los colores de las ves�culas en las im�genes, ya que se trata de dos enfermedades que presentan ves�culas muy similares. En trabajos futuros, se pueden realizar comparaciones contra otra enfermedad, como sarampi�n o herpes simplex.

Se presenta como un aspecto a investigar si las divergencias KLD entre una muestra de varicela o herpes y el histograma promedio responden a alg�n tipo de distribuci�n que permita realizar tests de intervalos de confianza y as� poder realizar la discriminaci�n.

Otro aspecto que puede trabajarse a futuro es el an�lisis de la distribuci�n de las ves�culas: al ser la varicela una enfermedad sist�mica, afecta a toda la piel de modo uniforme, pero el herpes z�ster se concentra agrupado en dermatomas, lo que permitir�a analizar la dispersi�n de las ves�culas para discriminar.

\chapter*{Agradecimientos}

\noindent Queremos dar las gracias a nuestras familias y amigos por el apoyo incondicional.

\include{07-referencias}
\paragraph{Datos de Contacto}

\emph{
\\Virginia In�s Arroyo
\\Universidad de Buenos Aires, Facultad de Ciencias Exactas y Naturales, Departamento de Computaci�n.
\\Pabell�n I, Ciudad Universitaria (C1428EGA), Buenos Aires, Argentina.
\\virginia.arroyo@gmail.com
}

\emph{
\\Juli�n Ricardo Oyola
\\Universidad de Buenos Aires, Facultad de Ciencias Exactas y Naturales, Departamento de Computaci�n.
\\Pabell�n I, Ciudad Universitaria (C1428EGA), Buenos Aires, Argentina.
\\joyola@dc.uba.ar
}


\end{document}
